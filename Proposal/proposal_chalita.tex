\documentclass[11pt,twoside]{article}

%%%%%%%%%%%%%%%%%%%%%%%%%%%%%%%%%%%%%%%%%%%%%%%%%%%%%%%%%%%%%%%%%%%%%%%%%%%%%

% Definitions for the title page
% Edit these to provide the correct information
% e.g. \newcommand{\reportauthor}{Timothy Kimber}

\newcommand{\reporttitle}{Evolutionary trajectories of antibiotic resistance genes in \textit{Burkholderia pseudomallei}}
\newcommand{\reportauthor}{Chalita Chomkatekaew}
\newcommand{\supervisor}{Prof. Timothy Barraclough}

%%%%%%%%%%%%%%%%%%%%%%%%%%%%%%%%%%%%%%%%%%%%%%%%%%%%%%%%%%%%%%%%%%%%%%%%%%%%%

% load some definitions and default packages
\input{includes}

\date{April 2022}

\setlength{\parindent}{0cm}

\begin{document}

% load title page
% Last modification: 2015-08-17 (Marc Deisenroth)
\begin{titlepage}

\newcommand{\HRule}{\rule{\linewidth}{0.5mm}} % Defines a new command for the horizontal lines, change thickness here


%----------------------------------------------------------------------------------------
%	LOGO SECTION
%----------------------------------------------------------------------------------------

\includegraphics[width = 4cm]{./figures/imperial}\\[0.5cm] 

\center % Center remainder of the page

%----------------------------------------------------------------------------------------
%	HEADING SECTIONS
%----------------------------------------------------------------------------------------

\textsc{\Large Imperial College London}\\[0.5cm] 
\textsc{\large Department of Life Science }\\[0.5cm] 

%----------------------------------------------------------------------------------------
%	TITLE SECTION
%----------------------------------------------------------------------------------------

\HRule \\[0.4cm]
{ \huge \bfseries \reporttitle}\\ % Title of your document
\HRule \\[1.5cm]
 
%----------------------------------------------------------------------------------------
%	AUTHOR SECTION
%----------------------------------------------------------------------------------------

\begin{minipage}{0.4\textwidth}
\begin{flushleft} \large
\emph{Author:}\\
\reportauthor
\end{flushleft}
\end{minipage}
~
\begin{minipage}{0.4\textwidth}
\begin{flushright} \large
\emph{Supervisor:} \\
\supervisor
\end{flushright}
\end{minipage}\\[4cm]


%----------------------------------------------------------------------------------------
%	FOOTER & DATE SECTION
%----------------------------------------------------------------------------------------
\vfill % Fill the rest of the page with whitespace
MSc Computational Methods in Ecology and Evolution 2021 \\

\small{Keywords: Bacteria, \textit{Burkholderia pseudomallei}, Antibiotics, Resistance, Evolutionary dyanmics, Antibiotics pollution}

\makeatletter
\@date 
\makeatother


\end{titlepage}


\newpage
\linenumbers
\textbf{\Large Introduction}\\

\begin{spacing}{1.5}


Burkholderia pseudomallei (Bp) is an environmental bacterium and can occupy a wide range of niches, including contaminated soil, and water. This bacterium is the causative agent of a neglected infectious disease called “Melioidosis”. The disease is predominately endemic in tropical and subtropical countries where the estimated mortality ranges from 10-40 \% of cases. Recent work suggested that Melioidosis is a global and a public health problem with the annual incident of the disease as high as 165,000 cases worldwide \citep{limmathurotsakul_predicted_2016}. Melioidosis exhibits similar manifestations to other diseases for instance bacterial infection and tuberculosis, and often leads to misdiagnosis. Given no vaccine is currently available, timely diagnostic and disease management are extremely important \citep{wiersinga_melioidosis_2018}.

\vspace{1 mm}

Antibiotic resistance is one of the major public health problems with efforts to elucidate the mechanisms of resistance in most well-known bacteria.  However, the mechanisms and evolutionary dynamics of antibiotic resistance in Bp are largely unknown. Given its environmental habitats, Bp is intrinsically resistant to common antibiotic secreted by soil microbes \citep{schweizer_mechanisms_2012}. This could be because of strong interspecific competition in Bp natural habitats. Antibiotics may principally act as agents of competitions against microbial competitors; such condition may have shaped the bacteria to become antibiotic resistance. In support of this hypothesis, my preliminary analyses suggest 1) most Antibiotic Resistant Genes (ARGs) in Bp are part of core-genome; 2) the resistant variants to the current treatment regime are distributed not only in the clinical isolates but also in the environmental ones; and 3) the molecular clock signals suggested that the resistant variants are likely to exist before the antibiotic usage in clinical settings. However, it is unclear how the bacterium evolution and environmental conditions, particularly, the antibiotic pollution play a role in the maintenance/selection of antibiotic resistance in Bp population. 

\vspace{1 mm}

Using computational analyses, I aim to evaluate the evolutionary dynamics of antibiotic resistance genes/variants in Bp population. This knowledge could help to improve the success of treatment and consequently the outcome of patients, particularly in low- and middle-incomes countries where supply and choices of antibiotics are limited.

\end{spacing}

\vspace{1 mm}

\textbf{\Large Proposed methods and data}\\

\begin{spacing}{1.5}
	Using a global collection of Bp whole-genome sequences (n=3341) and the associated metadata, I have previously characterised a comprehensive database of ARGs (n=194) in the collection. In attempt to elucidate the evolutionary dynamics of ARGs in Bp, the aims of this proposed project are follows:

\begin{itemize}
	\item \textbf{Aim 1:}The genetic variation of ARGs, particularly, single-nucleotide polymorphisms (SNPs) will be detected and compared between the lineages, sources, and countries to infer the diversity of ARGs. 
	\item \textbf{Aim 2:} The role of recombination in generating new ARG variants will be tested. 
	\item \textbf{Aim 3:} Selection pressure of ARGs will be quantified to infer the genes selective advantages in Bp.  Different modelling approaches are proposed, with PAML \citep{yang_paml_2007} and BayeScan \citep{foll_genome-scan_2008} methods. Each model will be evaluated for model selection. Observed patterns will be interpreted with respect to the role of each gene in antibiotic resistance mechanisms.
	\item \textbf{Aim 4:} If time permits, correlations between the positively selected ARGs and global antibiotic usages \citep{browne_global_2021} in the environment over time in each geographical location will be examined.
\end{itemize}

\end{spacing}

\vspace{1 mm}

\textbf{\Large Anticipated outputs and outcomes}\\
\begin{spacing}{1.5}
	The anticipated outcomes of the project are to comprehensively investigate the ARGs genetic variations and their frequencies in the Bp global population over time. The mode of ARGs dissemination will inform the evolutionary stability of ARGs in Bp.  Together, these approaches will allow me to identify the potential variants that could pose the threats to the current antibiotic regime strategies and determine appropriate antibiotic resistance surveillance strategy in Bp. Finally, while the correlation between the antibiotic usage and antibiotic resistance frequency may not inform the direct causation of resistance in Bp, the results will increase the awareness in the effects of antibiotic pollution to the environments and bacterial community. 
\end{spacing}

\vspace{1 mm}

\textbf{\Large Project timeline}\\

\begin{ganttchart}[expand chart=\textwidth,
	vgrid={draw=none, dotted},
	title/.style={fill=teal, draw=none},title label font=\color{white},
	title left shift=.1,
	title right shift=-.1,
	title top shift=.05,
	title height=.75,
	bar label node/.style={text width= 4 cm,
		align=right,
		anchor=east,
		font=\footnotesize\raggedleft}
	]{1}{24}
	\gantttitle{April}{4}
	\gantttitle{May}{4}
	\gantttitle{June}{4}
	\gantttitle{July}{4}
	\gantttitle{August}{4}
	\gantttitle{September}{4} \\
	\ganttbar{Data preparation }{1}{4} \ganttnewline
	\ganttbar{First draft introduction}{1}{4} \ganttnewline
    \ganttbar{ Aim 1. Genetic variations} {5}{8} \ganttnewline
    \ganttbar{ Aim 2. Recombination } {7}{10} \ganttnewline
    \ganttbar{ Aim 3. Selection pressure} {11}{14}\ganttnewline
    \ganttbar{ Aim 4. Correlation } {15}{16} \ganttnewline
    \ganttbar{Write up and viva preparation}{15}{24}
    
\end{ganttchart}

\vspace{1 mm}
\textbf{\Large Budget}

\begin{itemize}
	\item \pounds 125 (\pounds 25 per trip) Travel cost to Oxford for monthly update meeting.
\end{itemize}

\vspace{1 mm}
\textbf{\Large Supervisor approval}\\

\begin{spacing}{1.5}
	Hi Chalita,
	
	That looks good, nice job. Have a good weekend!
	
	best wishes,
	
	Tim \\
	(18$^{th}$ March 2022)
\end{spacing}

\bibliography{MSc-proposal}
%\printbibliography

\end{document}
