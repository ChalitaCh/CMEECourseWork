\documentclass[10pt]{article}
\usepackage{indentfirst}
\usepackage{graphicx}
\graphicspath{{../results}}
\usepackage{blindtext}
\usepackage{geometry}
 \geometry{
 a4paper,
 total={170mm,257mm},
 left=15mm,
 right=15mm,
 top=5mm,
 bottom=5mm
 }

\title{Is Florida getting warmer?}
\author{Chalita Chomkatekaew}
\date{31$^{st}$ October 2021}

\begin{document}

\maketitle
        
        Global warming is an gradula increase of overall earth's temperature due to human activities since the Pre-industrial period, primarily fossil fuel burning.
        The long term effects to the global climate change are various and interconnected, ranging from the increase in global surface temperature, more drought to the melting of the ice caps in the arctic.
        While there are collective efforts to reduce the effects of global warming, the consequences are irreversible, and will worsen in the decades to come.
        In this report, I investigated a dataset of annual temperature from Key West in Florida, USA collected over 10 years (Figure 1).
        We aimed to assess the correlation between the increased in temperature in Florida and time.
        
        \begin{figure}[h]
            \centering
            \includegraphics[scale = 0.6]{DistributionFlorida}
            \caption{Distribution of the annual temperatures in Florida during 1901-2000}
            \label{fig:DistributionFlorida1}
        \end{figure}

    The correlation coefficient test between the temperature measured each year was performed, giving R$^{2}$ of 0.53, suggesting moderate positive correlation.
    To account for the dependent relationship in sampling with successive time series, I performed a correlation coefficient test with 100000 permutations to assess the robustness of the test performed.
    The probability of the random correlation tests are higher or equal to the correlation coefficient observed gave the P-value of less than 0.05 (Figure 2).
    Despite the moderate magnitude of correlation, the relationship between the temperature and time in Florida from 1901 - 2000 is significant.
    
        \begin{figure}[h]
            \centering
            \includegraphics[scale = 0.6]{PermuCorCoeff_Florida}
            \caption{Distribution of the randomised correlation coefficient tests. The observed correlation coefficient is annotated in red dashed line}
            \label{fig:PermuCorCoeff_Florida1}
        \end{figure}

\end{document}
